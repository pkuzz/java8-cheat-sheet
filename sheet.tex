% source is based on template taken from http://stdout.org/~winston/latex/latexsheet.tex
\documentclass[10pt,landscape]{article}
\usepackage{multicol}
\usepackage{calc}
\usepackage{ifthen}
\usepackage[landscape]{geometry}
\usepackage{hyperref}
\usepackage{etoolbox}
\usepackage{array}

\ifthenelse{\lengthtest { \paperwidth = 11in}}
	{ \geometry{top=.5in,left=.5in,right=.5in,bottom=.5in} }
	{\ifthenelse{ \lengthtest{ \paperwidth = 297mm}}
		{\geometry{top=1cm,left=1cm,right=1cm,bottom=1cm} }
		{\geometry{top=1cm,left=1cm,right=1cm,bottom=1cm} }
	}

% Turn off header and footer
\pagestyle{empty}
 

% Redefine section commands to use less space
\makeatletter
\renewcommand{\section}{\@startsection{section}{1}{0mm}%
                                {-1ex plus -.5ex minus -.2ex}%
                                {0.5ex plus .2ex}%x
                                {\normalfont\large\bfseries}}
\renewcommand{\subsection}{\@startsection{subsection}{2}{0mm}%
                                {-1explus -.5ex minus -.2ex}%
                                {0.5ex plus .2ex}%
                                {\normalfont\normalsize\bfseries}}
\renewcommand{\subsubsection}{\@startsection{subsubsection}{3}{0mm}%
                                {-1ex plus -.5ex minus -.2ex}%
                                {1ex plus .2ex}%
                                {\normalfont\small\bfseries}}

%\preto{\@verbatim}{\topsep=0pt \partopsep=0pt }

\makeatother

% Define BibTeX command
\def\BibTeX{{\rm B\kern-.05em{\sc i\kern-.025em b}\kern-.08em
    T\kern-.1667em\lower.7ex\hbox{E}\kern-.125emX}}

% Don't print section numbers
\setcounter{secnumdepth}{0}


\setlength{\parindent}{0pt}
\setlength{\parskip}{0pt plus 0.5ex}


% -----------------------------------------------------------------------

\begin{document}

\raggedright
\footnotesize
\begin{multicols}{2}


% multicol parameters
% These lengths are set only within the two main columns
%\setlength{\columnseprule}{0.25pt}
\setlength{\premulticols}{1pt}
\setlength{\postmulticols}{1pt}
\setlength{\multicolsep}{1pt}
\setlength{\columnsep}{2pt}

\begin{center}
     {\Large{\textbf{Java 8 Cheat Sheet}}} \\~\\ 
Made after reading Java SE 8 for the Really Impatient \\ by Cay~S.~Horstmann
\end{center}

\newlength{\DefaultColLen}
\settowidth{\DefaultColLen}{.................................................}




\section{General Info}
\subsection{Functional Interfaces}
\begin{itemize}
\setlength\itemsep{0px}
\item every interface with a single abstract method is a functional interface
\item use \verb!@FunctionalInterface! where possible
\item lambda's checked exceptions must match method's exceptions
\item use \verb!Default Methods! to extends functionality
\item interfaces can have static methods
\end{itemize}

\subsection{Default Methods - Diamond problem}
\begin{itemize}
\setlength\itemsep{0px}
\item Superclasses win. Concrete method of superclass has higher priority than any interface methods
\item Interfaces clash. Explicitly select which interface method should be called.
\end{itemize}

\subsection{JDK Common Functional Interfaces}
\begin{tabular}{@{}m{\the\DefaultColLen}@{}m{\linewidth-\the\DefaultColLen}@{}}
\verb!Runnable! & simply run action, no params, no return value \\
\verb!Supplier<T>! & no params, creates value of type \verb!T! \\
\verb!Consumer<T>! & consumes single param of type \verb!T!, no return value \\
\verb!BiConsumer<T, U>! & consumes two values of types \verb!T! and \verb!U!, no return value \\
\verb!Function<T, R>! & function of value of type \verb!T! that return value of type \verb!R! \\
\verb!BiFunction<T, U, R>! & function of values of types \verb!T! and \verb!U! that return value of type \verb!R! \\
\verb!UnaryOperator<T! & function of value of type \verb!T! that return value of type \verb!T! \\
\verb!BinaryOperator<T>! & function of two values of type \verb!T! that return value of type \verb!T! \\
\verb!Predicate<T>! & boolean-valued function of value of type \verb!T! \\
\verb!BiPredicate<T, U>! & boolean-valued function of two values of type \verb!T! and \verb!U! \\
\verb!! &  \\
\end{tabular}

\subsection{Optional}
\begin{tabular}{@{}m{\the\DefaultColLen}@{}m{\linewidth-\the\DefaultColLen}@{}}
\verb!???! & ???\\
\end{tabular}





\section{Lambda Expressions}

\subsection{Syntax}
\begin{tabular}{@{}m{\the\DefaultColLen}@{}m{\linewidth-\the\DefaultColLen}@{}}
\verb!() -> statement! & single statement with no params \\
\begin{minipage}[t]{\linewidth}
\begin{verbatim}
() -> {
    body
}
\end{verbatim}
\end{minipage} & multi-line code block with no params\\
\verb!(T a1, U a2) -> statement! & single statement with two params \\
\verb!(a1, a2) -> statement! & statement with two params with inferred types\\
\verb!() -> { return 1; }! & it is illegal to return from lambda\\
\end{tabular}

\subsection{Method and Constructor References}
\begin{tabular}{@{}m{\the\DefaultColLen}@{}m{\linewidth-\the\DefaultColLen}@{}}
\verb!object::instanceMethod! & 
	reference to an instance's method, shortcut to 
	\verb!(ps) -> object.instanceMethod(ps)! \\
\verb!Class::staticMethod! & 
	reference to a static method of class, shortcut to 
	\verb!(ps) -> Clazz.staticMethod(ps)! \\
\verb!Class::instanceMethod! & 
	reference to a first parameter's method, shortcut to 
	\verb!(x, ps) -> x.instanceMethod(ps)!  where \verb!x! is of type \verb!Class! \\
\verb!this::instanceMethod! & 
	reference to a method of enclosing instance, shortcut to
	\verb!(ps) -> this.methodInstance(ps)! \\
\verb!super::instanceMethod! & 
	reference to a method of parent of enclosing instance, shortcut to
	\verb!(ps) -> super.methodInstance(ps)! \\
\verb!Class::new! &
	constructor reference, if there multiple constructors, compiler tries to determine which one to use from context \\
\verb!type[]::new! & 
	array's constructor with given length, useful in some \verb!toArray! methods:
	\verb!stream.toArray(Button[]::new)!
\end{tabular}





\section{Stream API}

\subsection{Summary}
\begin{itemize}
\setlength\itemsep{0px}
\item streams may be finite or not
\item streams does not store theirs elements
\item stream operations never mutate the source
\item stream operations are lazy where possible
\item there tow kinds of stream operations - \textit{intermediate} (transforming stream) and \textit{terminal} (producing result, after that stream is no longer usable)
\item stream is \verb!AutoCloseable!
\end{itemize}

\subsection{Stream Creation}
\begin{tabular}{@{}m{\the\DefaultColLen}@{}m{\linewidth-\the\DefaultColLen}@{}}
\verb!Stream.of(array)! & stream from array, generic type is inferred\\
\verb!Stream.of(v1, v2, v3, ...)! & stream from specified values, generic type is inferred\\ 
\verb!Array.stream(ary, from, to)! & stream from part of and array \\
\verb!Stream.empty()! & empty stream, generic type is inferred\\
\verb!Stream.generate(supplier)! & generates infinite stream with values from \verb!supplier! \\
\verb!Stream.iterate(seed, op)! & generates infinite stream with values started with \verb!seed! creating next values by \verb!UnaryOperator op! function 
\end{tabular}

\subsection{Stream Transformation}
\begin{tabular}{@{}m{\the\DefaultColLen}@{}m{\linewidth-\the\DefaultColLen}@{}}
\verb!filter(p), map(f), flatMap(f)! & basic transformations\\
\verb!skip(n), limit(n)! & new stream with skipped given count of entries, or taken given count of entities \\
\verb!concat(s1, s2)! & concatenates two streams, first stream should be finite to get a change for a second one\\
\verb!peek(consumer)! & new stream with the same items, but \verb!consumer! applied for each item, useful for debugging\\
\verb!distinct()! & stateful, suppress duplicates\\
\verb!sorted(comparator)! & stateful, return new sorted stream
\end{tabular}

\subsection{Stream Reductions}
\verb!reduce!, \verb!anyMatch!, \verb!allMatch!, \verb!noneMatch!, \verb!findAny!, \verb!findFirst!, \verb!min!, \verb!max!, \verb!sum!, etc...         

\subsection{Collecting Results}
\begin{itemize}
\setlength\itemsep{0px}
\item collecting to simple collections: \verb!Collectors.toSet()!, \verb!Collectors.toList()!, \verb!Collectors.toCollection(collection factory)!, ...
\item collecting streams: \verb!Collectors.joining()!, ...
\item collecting numeric: \verb!Collectors.summarizing[Int|Long|Double]()!, ...
\item terminal iteration over stream: \verb!stream.forEach(consumer)!, \verb!stream.forEachOrdered(consumer)!
\item collecting to map: \verb!Collectors.toMap(keyFunc, valueFunc[, mergeBinaryOp[, mapSupplier]])!
\item grouping and partitioning: \verb!Collectors.groupBy(classifierFunction)!, \verb!Collectors.partitioningBy(predicate)!
\item grouping with downstream collectors: \verb!Collectors.groupBy(classifierFunction, downstreamCollector)!, don't forget, downstream collectors may be deeply nested
\end{itemize}

\subsection{Parallel Streams}
\begin{tabular}{@{}m{\the\DefaultColLen}@{}m{\linewidth-\the\DefaultColLen}@{}}
\verb!???! & ???\\
\end{tabular}


Copyright \copyright\ 2015 Pavel Kuzmenko


\end{multicols}
\end{document}